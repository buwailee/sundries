%!TEX program = xelatex
\documentclass[10pt]{extarticle}
\usepackage[zh]{noteheader}
\usepackage{ctex}
\pagestyle{empty}
\newcommand{\no}[1]{{$(#1)$}}%编号
\begin{document}
\begin{center}
	2015年南京大学计算物理期末考试卷(由Unsinn默写)
\end{center}

\noindent 1.现有时间-速度表:

\begin{table}[h]
\centering
\begin{tabular}{l c c c c c c}
\hline
$t(\mathrm{s})$ & 0 & 10 & 15 &20 &23 &30\\
$v(\mathrm{m/s})$ & 0 & 230 & 263 & 518 &603 &902\\
\hline
\end{tabular}
\end{table}
\noindent $(1)$. 用抛物线(三点插值)求$t=16$时候的速度。\\
$(2)$. 使用最小二乘法拟合曲线,拟合函数为$f(x)=ax^2+bx+c$,求$a,b,c$.

\vspace{10ex}

\noindent 2.写出打靶法求方程
\[
	\begin{cases}
		\displaystyle{\frac{\mathrm{d}^2\varphi}{\mathrm{d}x^2}=-k^2 \varphi};\\
		\varphi(0)=\varphi(1)=0,
	\end{cases}
\]
的最小本征值的(C或Fortran)程序。

\vspace{10ex}

\noindent 3.在抛物线势能的势阱中的粒子初始波函数为
\[
	\Psi(x,0)=\sqrt{\frac{a}{\pi}}\exp\bigl(-ax^2\bigr),
\]
满足方程
\[
	i\hbar \frac{\partial \Psi}{\partial t}=-\frac{\hbar^2}{2m}\frac{\partial^2 \Psi}{\partial x^2}+b x^2\Psi(x,t).
\]
写出求解随时间演化的波函数的(C或Fortran)程序。

\vspace{10ex}

\noindent 4.写出一种Monte-Carlo方法求$\pi$的(C或Fortran)程序。

\vspace{10ex}

\noindent 5.在分子动力学模拟中\\
$(1)$.写出一种积分运动方程的算法及其优缺点;\\
$(2)$.在等温单原子气体的模拟中,写出初始化粒子速度的(C或Fortran)程序。
\end{document}